\documentclass[11pt]{article}
\usepackage{amsmath}
\usepackage{amssymb}
\usepackage[margin=.5in,left=.5in]{geometry} 
\usepackage{amsthm}
\usepackage{pgfplots}
\pgfplotsset{compat=1.18}
\setlength{\parindent}{0pt}

\begin{document}

\begin{titlepage}
    \centering
    \vspace*{\fill} % Push content to vertical center
    {\Large An Introduction to Mechanics} \\[0.5cm]
    {\Large by Daniel Klepper \& Robert Kolenkow} \\[0.5cm]
    {\large Victor C. Van} \\[1cm]
    {\large December 27, 2024}
    \vspace*{\fill} % Push content to vertical center
\end{titlepage}

\newpage

\begin{center}
    \item \section*{Chapter 1: Vectors \& Kinematics}
\end{center}

\begin{center}
    \item \subsection*{Chapter 1.1: Introduction}
\end{center}

Mechanics is at the heart of physics; its concepts are essential for understanding the world around us and phenomena on scales from atomic to cosmic.
Concepts such as momentum, angular momentum, and energy  play roles in practically every part of physics.
The goal of this bookis to help you acquire a deep understanding of the principles off mechanics.\\\\

The reason we start by discussing vectors and kinematics rather than plunging into dynamics is that we want to use
these tools freely in discussing physical principles. Rather than interrupt the flow of discussion later,
we are taking the timne now to ensure they are on hand when requested.

\begin{center}
    \item \subsection*{Chapter 1.2: Vectors}
\end{center}

Our principle motivation for inroducting vectors is to simplify the form of equations.
However, as we shall see in chapter 14, vectors have a much deeper significance. Vectors are closely related to the
fundamental ideas of symmetry and their use can lead to valuable insights into the possible forms of unknown laws. 

\begin{center}
    \item \subsubsection*{Chapter 1.4.1: Dot Product} 
\end{center}

\[
    \vec{A} \cdot \vec{B} \equiv |\vec{A}||\vec{B}|cos(\theta)
\]

\textbf{The Law of Cosines}
The law of cosines relates the lengths of three sides of a triangle to the cosine of one of its angles.
Following the notation of the drawing, the law of cosines insights

\[
    C^{2} = A^{2} + B^{2} - 2ABcos{\phi}
\]

The law can be proved by a variety of trigonmetric or geometric constructions, but none is so simple and elegent
as the vector proof, which merely involves squaring the sum of the two vectors.

\[
    \vec{C} = \vec{A} + \vec{B}
\]

\[
    \vec{C} \cdot \vec{C} = (\vec{A} + \vec{B}) \cdot (\vec{A} + \vec{B}) = \vec{A} \cdot \vec{A} + \vec{B} \cdot \vec{B} + 2(\vec{A} \cdot \vec{B}) 
\]

\[
    |\vec{C}|^{2} = |\vec{A}|^{2} + |\vec{B}|^{2} + 2|\vec{A}||\vec{B}|cos(\theta)
\]

Recognizing that $cos(\phi)$ = -$cos(\theta)$ completes the proof.

\newpage

\begin{center}
    \item \subsubsection*{Chapter 1.4.2: Cross Product} 
\end{center}

\[
    \vec{C} = \vec{A} \times \vec{B} = |\vec{A}||\vec{B}|sin(\theta)
\]

\[
    \vec{B} \times \vec{A} = -\vec{A} \times \vec{B}
\]

\end{document}