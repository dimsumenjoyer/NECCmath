\documentclass{article}
\usepackage{amsmath}
\usepackage{amssymb}
\usepackage[margin=.5in,left=.5in]{geometry}
\usepackage{amsthm}
\renewcommand{\qedsymbol}{QED}
\setlength{\parindent}{0pt}

\begin{document}

\begin{titlepage}
    \centering
    \vspace*{\fill} % Push content to vertical center
    {\Large Calculus 3 Proof} \\[0.5cm]
    {\Large (jokingly called "The Victor Theorem", according to a classmate)} \\[0.5cm]
    {\Large Written for: Professor Habib Maagoul} \\[0.5cm]
    {\large Victor C. Van} \\[1cm]
    {\large November 27, 2024}
    \vspace*{\fill} % Push content to vertical center
\end{titlepage}

\begin{proof}

For simplicity, we can start with a basic double integral. This principle can be applied to any arbitrary n-number of iterated integrals.
\[
    \iint_Rf(x,y) \, dA.
\]

For any iterated integral of some function $f(x,y)$ in any Euclidean coordinate system (e.g. Cartesian, polar, and cylindrical), such that $f(x,y)$ can be expressed as the product of two independent functions $g(x)$ and $h(y)$, and that the bounds of integration are constants or infinities:

\[
    \int_{c}^{d} \int_{a}^{b} f(x, y) \, dA = \int_{c}^{d} \int_{a}^{b} g(x)h(y) \,dxdy
\]

\[
    \{(a, b, c, d) \in \mathbb{R} \mid - \infty \le (a, b, c, d) \le \infty \}
\]

\[
    h(y) \text{ is a constant when you integrate with respect to } x.
\]

\[
    g(x) \text{ is a constant when you integrate with respect to } y.
\]

We can represent this as the product of two definite integrals of functions that are independent of each other:

\[
\int_{a}^{b}g(x) \,dx \cdot \int_{c}^{d}h(y) \,dy
\]

If this claim is true, it must also apply to partial derivatives due to the \textit{Fundamental Theorem of Calculus}.\\

Suppose some function $f(x, y) = x^2 y^3$.

\[
    f(x, y) = g(x) \cdot h(y); \quad g(x) = x^2, \quad h(y) = y^3
\]

\[
    \frac{\partial^{2} f}{\partial x \partial y} = \frac{\partial^{2}f}{\partial y \partial x} = 6xy^{2}
\]

\[\frac{\partial g}{\partial x} = 2x\]
\[\frac{\partial h}{\partial y} = 3y^{2}\]
\[\frac{\partial g}{\partial x} \cdot \frac{\partial h}{\partial y} = 6xy^{2}\]
Thus:

\[
    \frac{\partial^2 f}{\partial x \partial y} = \frac{\partial^2 f}{\partial y \partial x} = \frac{\partial g}{\partial x} \cdot \frac{\partial h}{\partial y} = \frac{dg}{dx} \cdot \frac{dh}{dy}
\]

Therefore, if and only if $f_i(x_i)$ is independent from $f_{i+1}(x_{i+1})$
\[
\prod_{i=1}^n \int_{a_i}^{b_i} f_i(x_i) \, dx_i = \int_{a_1}^{b_1} f_1(x_1) \, dx_1 \cdot \int_{a_2}^{b_2} f_2(x_2) \, dx_2 \cdots \int_{a_n}^{b_n} f_n(x_n) \, dx_n
\]

\[
    \{(a, b, c, d) \in \mathbb{R} \mid - \infty \le (a, b, c, d) \le \infty \ \mid i \in \mathbb{N}\}
\]

The proof is trivial.

\end{proof}

\newpage

\textbf{Fubini's Theorem:}

The triple integral of a continious function \( f(x, y, z) \) over a box \( B \), where $a \leq x \leq b$, $c \leq y \leq d,$ and $e \leq z \leq f$ to the integral:
\[
    \iiint_B f(x, y, z) \, dV = \int_{a}^{b} \int_{c}^{d} \int_{e}^{f}f(x,y,z)dzdydx
\]

Below is in-class example that this proof was created for, and my work. It was contested if this method of iterated integration was mathematically correct.
I was unable to find a proof that sufficed to prove my claim online; therefore, this became my motivation to create a proof myself.
Take note about how my proof is basically just a special case of \textit{Fubini's Theorem} when  $f_1(x_1)$, $f_{2}(x_{2})$, and $f_{3}(x_{3})$ are all independent of each other.\\

\textbf{Evaluate $\iiint_B xyz^{2} \, dV$ where $ B = \{(x,y,z) \mid  0 \leq x \leq 1, -1 \leq y \leq 2, 0 \leq z \leq 3\}$}

\[
    \int_{0}^{1} \int_{-1}^{2} \int_{0}^{3}(xyz^{2}) \, dzdydx
\]

\[
    = \int_{0}^{1}x \, dx \cdot \int_{-1}^{2}y \, dy \cdot \int_{0}^{3}z^{2} \, dz
\]

\[
    = \frac{1}{2}[x^{2}]_{0}^{1} \cdot \frac{1}{2}[y^{2}]_{-1}^{2} \cdot \frac{1}{3}[z^{3}]_{0}^{3}
\]

\[
    = \frac{1}{12}[x^{2}]_{0}^{1} \cdot [y^{2}]_{-1}^{2} \cdot [z^{3}]_{0}^{3}
\]

\[
    = \frac{3^{3}}{12}((2)^{2}-(-1)^{2})
\]

\[
    = \frac{3^{3}}{12}(4-1)
\]


\[
    = \frac{27}{4}
\] 

\end{document}